%%% $Id$
%%% Copyright (C) 2002 Greg J. Badros <greg.badros@infospace.com>
%%
%
% 30 word statement of contribution:
%
% ETL demonstrates that web templating does not require a full
% programming language. A restricted XML-based language improves
% analyzability, potentially reducing costs and increasing quality by
% eliminating whole classes of errors.
%
% This file should be compiled with V1.0 of "www2003-submission.cls"
%
% ----------------------------------------------------------------------------------------------------------------
% This .tex file (and associated .cls V1.0) produces:
%       1) NO Permission Statement
%       2) WWW'03-specific conference (location) information
%       3) The Copyright Line with ACM data
%       4) NO page numbers
%
% ---------------------------------------------------------------------------------------------------------------
% This .tex source is an example which *does* use
% the .bib file (from which the .bbl file % is produced).
% REMEMBER HOWEVER: After having produced the .bbl file,
% and prior to final submission, you *NEED* to 'insert'
% your .bbl file into your source .tex file so as to provide
% ONE 'self-contained' source file.
%

\documentclass{www2003-submission}
\usepackage{times}
\usepackage{moreverb}
\usepackage{floatflt}
\usepackage{url}

\newcommand{\B}{\discretionary{}{}{}}
\newcommand{\smtt}{\small}
\newcommand{\smtexttt}[1]{{\small\texttt{#1}}}
\newcommand{\ns}[1]{{\small\texttt{#1:*}}}
\newcommand{\figref}[1]{Fig.~\ref{fig-#1}}
\newcommand{\secref}[1]{Section~\ref{sec-#1}}
\newcommand{\ssecref}[1]{Section~\ref{ssec-#1}}
\newcommand{\tableref}[1]{Table~\ref{#1}}
\newcommand{\tm}{{\scriptsize $^{\mbox{tm}}$}}
\newcommand{\gjb}[1]{{\sc gjb:}\textbf{#1}}
%\newcommand{\etl}{\textsf{ETL}}
%\newcommand{\etls}{\textsf{ETLS}}
\newcommand{\etl}{ETL}
\newcommand{\etls}{ETLS}

\newenvironment{smallverbatim}%
{\renewcommand{\baselinestretch}{1}\small\verbatim}%
{\renewcommand{\baselinestretch}{2}\endverbatim}

\newenvironment{smalllisting}%
{\renewcommand{\baselinestretch}{1}\small\listing}%
{\renewcommand{\baselinestretch}{2}\endlisting}

\begin{document}
%
\title{The Extensible Templating Language: \\
       An XML-based Restricted Markup-Generating Language\vspace*{-.2in}}

\numberofauthors{1}

\author{
%
% The command \alignauthor (no curly braces needed) should
% precede each author name, affiliation/snail-mail address and
% e-mail address. Additionally, tag each line of
% affiliation/address with \affaddr, and tag the
%% e-mail address with \email.
\alignauthor Greg J. Badros, Abhishek Parmaar\\
       \affaddr{InfoSpace, Inc., 601 108th Ave. NE, Suite 1200, Bellevue, WA 98004, USA}\\
       \email{badros@cs.washington.edu, aparmar@infospace.com}}

\date{17 February 2003}
\maketitle
\vspace*{-.7in}
\begin{abstract}
{ \small Unlike most popular web templating languages, the Extensible
Templating Language does not embed a general-purpose programming
language. Instead \etl{} restricts the set of language features, hence
limiting what can be done by templates and ensuring a better
separation of presentation from business logic. Furthermore,
\etl{} makes source templates easier to analyze by using an
XML-based representation where markup is intermingled with XML
elements corresponding to programming constructs. This approach helps
ensure proper markup and facilitates tool-building. \etl{} runs inside
of InfoSpace's ETL Server which serves millions of requests per
day. (A more thorough discussion of ETL is available
online.~\cite{ETL-full}) }


\end{abstract}

% A category with only the three required fields
%% See http://www.acm.org/class/1998/overview.html
%\category{D.2}{Software}{Software Engineering}
%\category{D.3}{Software}{Prog.\ Languages}
%\category{C.2}{Computer Systems Organization}{Networks}

% GREGB:FIXME:: any terms?
%\terms{}

\keywords{Restricted domain-specific programming language, static analysis,
web server, templates, XML, XSLT, markup, WML, HTML\@.}

\section{Introduction}
\label{sec-intro}

Server-side markup generation is dominated by expressive and
flexible general-purpose programming languages such as Java, Perl, and
PHP\@.  Ironically, the languages are then used in fairly
restricted ways. For example, consider the PHP template in
\figref{php-books}.  The only constructs required by the template are
a) some means of populating a data model from the back-end business
logic (line 1); b) data-driven iteration (line 4); and c) simple
expression evaluation to access parts of the data model (lines 6 and
7). The extra power provided by typical templating languages not only
goes unused, but it also complicates analyses of the
code such as proving that processing terminates.

% GREGB:FIXME:: check the below line numbers 

\begin{figure}[ht]
\begin{smalllisting}{1}
<? $books = GetBooks(...); ?>
<html>
 <table>
  <? for ($i=0; $i < count($books); ++$i) { ?>
   <tr>
    <td><? print($books[$i][author]); ?> </td>
    <td><? print($books[$i][title]); ?> </td>
   </tr>
  <? } ?>
 </table>
</htm>
\end{smalllisting}
%$
\caption{Cleanly-written PHP templates require only the simplest
programming language constructs.\label{fig-php-books}}
\end{figure}

Instead of embedding a general-purpose programming language, the
Extensible Templating Language\footnote{This research is supported by
InfoSpace's advanced server development group.  \etl{} and \etls{} are
trademarks of InfoSpace, Inc.} is a domain-specific language
specifically tailored to the needs of generating markup.  This
approach limits the capabilities of the front-end and helps separate
front-end presentation from back-end application logic.  While other
languages might use an object-based data model to communicate between
tiers, ETL instead interacts only with web-services and exposes only
an XML data model.

Another problem with existing templating technologies is their mixing
of two separate programming paradigms and syntaxes.  JSP, for example,
is implemented in terms of a pre-processing rewrite into a Java
servlet. The precise analysis of an arbitrary JSP template necessarily
requires full knowledge of both the rewrite rules and the semantics of
Java.  Worse, templating languages generally are ignorant of the rules
of the markup being generated: the mistaken close tag on line 11 of
\figref{php-books} goes unnoticed by PHP, yet will break under an XHTML browser.

To combat the above problem, ETL directly intermingles literal markup
elements and XML-based representations of programming-language
constructs are intermingled directly.  By using XML as the surface
syntax, standard XML tools such as XSLT  can be used to perform
analyses and transformations~\cite{Badros-www9}.  An example \etl{}
template appears in \figref{etl-books} and is analogous to the PHP
example from
\figref{php-books}.

\begin{figure}[tb]
\begin{smalllisting}{1}
<?xml version="1.0"?>
<bl:template xmlns:bl="http://www....">
 <bl:set var="#xml/books">...</bl:set>
 <html>
  <table>
   <bl:for-each var="#xml/books/book">
    <tr> 
     <td><bl:get var="@author"/></td>
     <td><bl:get var="@title"/></td>
    </tr>
   </bl:for-each>
  </table>
 </html>
</bl:template>
\end{smalllisting}
\caption{The \etl{} template corresponding to \figref{php-books}'s PHP
code is required to be well-formed XML\@.  Thus, the developer
avoids the mistake from line 11 of the PHP example.
\label{fig-etl-books}}
\end{figure}

\section{\hspace*{-.1in}Extensible Templating Language}
\label{sec-etl}

An \etl{} template must be well-formed XML that is valid with respect
to the language's XML Schema Definitions.  This requirement catches a
large class of programming errors very early in the development
process without need for the \etl{} compiler at all.  Those schemas
are also used to support syntax coloring and completion of element and
attribute names.  \etl{} templates are processed within the context of
an HTTP request and have the primary goal of generating an HTTP
response (including the document and headers).

\etl{} supports dynamic behaviours by means of \emph{primitives} which
are elements in the \ns{bl} namespace.  Primitives either output text
to the current destination stream (initially the response document) or
perform some side-effect (or both).  For example the \smtexttt{bl:set}
primitive assigns a variable a value without outputting any text,
while the \smtexttt{bl:http} primitive outputs either \smtexttt{http}
or \smtexttt{https} based on whether the current request is secure.
The behaviour of a primitive is controlled by its attributes.  Some
primitives (e.g., \smtexttt{http}) are always empty while others are
allowed to be non-empty and control the contained elements.
For example, in:

\begin{smallverbatim}
<bl:if var="showcopyright"> (C) 2002 </bl:if>
\end{smallverbatim}

\noindent the contents of the \smtexttt{bl:if} are output if and
only if the variable's value is non-empty.

Many primitives derive their arguments not from individual attributes
but from pairs of attributes that are together called a \emph{slot}.
For example, the \smtexttt{bl:cr} directive outputs a number of
linefeeds determined by a slot called \smtexttt{count}.  That slot is
specified using a pair of mutually-exclusive attributes: \smtexttt{count} and
\smtexttt{count-var}.  The \smtexttt{count} attribute has type
\smtexttt{xsd:integer} and is used to provide a literal integer
argument to the primitive.  In contrast, the \smtexttt{count-var}
attribute has type \smtexttt{bl:identifierType} (a restriction of
\smtexttt{xsd:string}) and names a variable to evaluate at runtime.
In both cases, the resulting value is used as the number of linefeeds
to generate, but for \smtexttt{@count} that number is set statically
while for \smtexttt{@count-var} it is determined dynamically.

As with all imperative programming languages, \etl{} has the notion of
variables that store values. Variables are assigned values using the
\smtexttt{bl:set} primitive: the value is determined by executing the
contained elements. For example:

\begin{smallverbatim}
<bl:set var="baseuri"><bl:http/>://<bl:get
 var="hostname"/>/</bl:set>
\end{smallverbatim}

After assignment, values are retrieved by referencing them by name in
a slot or by copying them directly to the output stream using
\smtexttt{bl:get}.  There are numerous special reserved variables that may have
side-effects and convey request parameters
or other system-level details such as \smtexttt{\#browsertype}.
Various other data are accessible to \etl{} templates via
\emph{buckets}. Buckets replace field references and accessor methods of
distinguished objects in languages such as Java.  For example, where a
JSP developer might write \smtexttt{<\%= request.getHeader("User-Agent") \%>}
the \etl{} developer accesses the \smtexttt{\#http}
bucket instead:

\begin{smallverbatim}
<bl:get var="#http/User-Agent"/>
\end{smallverbatim}

A \emph{transformer} is a streaming converter from one byte sequence
to another.  For example, the \smtexttt{url-encode} transformer
converts ``hello world'' into ``hello+world''.  Transformers can be
used whenever a variable is being set (via
\smtexttt{bl:set}) or accessed (via \smtexttt{bl:get}) using the
\smtexttt{transform} attribute which specifies an ordered whitespace-separated
list of transformers to apply.  For example:

\begin{smallverbatim}
<bl:get var="a" transform="trim urlencode"/>
\end{smallverbatim}

\noindent results in writing out the value of the variable \smtexttt{a}
after first eliminating leading and trailing whitespace and then
URL-encoding the resulting trimmed string (the order of
transformations does matter).

There are five primary means of control flow in \etl{}: 1)
\emph{template selection} dispatches to the most specific template on
the basis of the current cobrand; 2) \emph{conditionals} include
\smtexttt{bl:if} and \smtexttt{bl:choose}/\B{}\smtexttt{bl:when}; 3)
\emph{loops} are only allowed to be data-driven; 4) \emph{exceptions} 
transfer control non-locally; and 5) \emph{remote invocations} allow
calling other web services as subroutines.

\emph{Literal markup elements} are copied directly to the response
stream from the source template.  One important common need is to
conditionally include a tag which \etl{} supports using a special
\smtexttt{bl:if-var} attribute on an arbitrary literal result element:

\begin{smallverbatim}
<b bl:if-var="#pro/wantbold">text</b>
\end{smallverbatim}

\noindent where the output of \emph{both} the open and close
\smtexttt{b} tags is controlled by the test of the single variable
(and the contents of the \smtexttt{b} element are always
executed). This technique also lets us factor out the guard, thus
eliminating the possibility of a mismatch between the test used for
the start tag and the one used for the end tag.

\etl{} permits fine-grained control over the \emph{whitespace} that will be
generated via two orthogonal special attributes that are allowed on
every XML element in source \etl{} documents:
\smtexttt{@bl:space-nodes} and \smtexttt{@bl:text-nodes}.  These
attributes are inherited by child elements and their contents unless
overridden.

\emph{custom tags} define new abstractions in terms of 
built-in primitives using an XSLT transformation rule.  Those
rules are then applied to subtrees of the source template and rewrite
that code to contain only the core language.  This approach allows a
great deal of flexibility in new abstractions without compromising our
tight control over the limited behaviour we permit the presentation
layer.

A byte-code compiler and runtime for the Extensible Templating
Language is an integral part of InfoSpace's \etl{} Server 2.0,
an ISAPI extension to IIS.  The server supports configuration 
via special \smtexttt{admin} primitives, has a powerful
HTTP-based reflection interface, and understands various special
\smtexttt{etl-*} URL parameters that help with debugging.
InfoSpace's production web tree contains over sixty thousand \etl{}
templates and the clusters of \etl{} Servers serve millions of
requests per day.

%%% 
\section{Related work \& conclusion}
\label{sec-related-work}

LAML has similar goals to \etl{}, but takes an opposite approach: LAML
embeds the simpler markup languages in Scheme, an especially powerful
and expressive language.  \verb|<bigwig>| is another significant
research project that has similar goals.  That work adds a first class
markup-fragment type to a conventional programming language.  That
system is more aggressive in its verification of the well-formedness
of markup values: an iterative data-flow analysis bounds the possible
values which can then be confirmed to be valid.

The Extensible Templating Language leverages a domain-specific
XML-based programming language intermingled with literal markup.  By
restricting the language, we preserve a better separation of front-
and back-ends. By using XML, we simplify analyses, transformations,
and make it easier to build supporting tools.  The static checks ETLS
performs reduce the need for testing and make development faster and
less error-prone.

\bibliographystyle{abbrv}
\bibliography{etl-www2003}
\balancecolumns

%\appendix
%Appendix A
%\section{}
%\balancecolumns % GM July 2000

\end{document}
