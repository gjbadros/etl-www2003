%%% $Id$
%%% Copyright (C) 2002 Greg J. Badros <greg.badros@infospace.com>
%%
%
% This file should be compiled with V1.0 of "www2003-submission.cls"
%
% ----------------------------------------------------------------------------------------------------------------
% This .tex file (and associated .cls V1.0) produces:
%       1) NO Permission Statement
%       2) WWW'03-specific conference (location) information
%       3) The Copyright Line with ACM data
%       4) NO page numbers
%
% ---------------------------------------------------------------------------------------------------------------
% This .tex source is an example which *does* use
% the .bib file (from which the .bbl file % is produced).
% REMEMBER HOWEVER: After having produced the .bbl file,
% and prior to final submission, you *NEED* to 'insert'
% your .bbl file into your source .tex file so as to provide
% ONE 'self-contained' source file.
%

\documentclass{www2003-submission}
\usepackage{times}
\begin{document}
%
\title{The Extensible Templating Language: \\
       Improving Static Analysis of Markup Generation}

\numberofauthors{1}

\author{
%
% The command \alignauthor (no curly braces needed) should
% precede each author name, affiliation/snail-mail address and
% e-mail address. Additionally, tag each line of
% affiliation/address with \affaddr, and tag the
%% e-mail address with \email.
\alignauthor Greg J. Badros\\
       \affaddr{InfoSpace, Inc.}\\
       \affaddr{601 108th Ave. NE, Suite 1200}\\
       \affaddr{Bellevue, WA ~ USA}\\
       \email{greg.badros@infospace.com}
}
\date{10 October 2002}
\maketitle
\begin{abstract}
ETL....
\end{abstract}

% A category with only the three required fields
%% GREGB:FIXME::
\category{D.2}{Software}{Software Engineering}
\category{D.3}{Software}{Programming Languages}

% GREGB:FIXME:: any terms?
%\terms{}

\keywords{Static analysis, web server, templates, XML, XSLT, HTML.}

\section{Introduction}

\subsection{Internet-speed development vs.\ maintainability}

\subsection{Flexibility vs.\ analyzability}

\subsection{Outline of paper}


%%% 
\section{Background}

\subsection{Server-side web applications}

\subsection{Current templating strategies}

\subsection{Comparison of approaches}

%\begin{table}
%\centering
%\caption{Summary of existing templating approaches.}
%\begin{tabular}{|c|c|l|} \hline
%\\ \hline
%\hline\end{tabular}
%\end{table}


%%% 
\section{ETL: A better approach}


%%% 
\section{Benefits of ETL}

%%% 
\section{Implementation and experience}

%%% 
\section{Related work}


%%% 
\section{Future work}


%%% 
\section{Conclusion}

%ACKNOWLEDGMENTS are optional
\section{Acknowledgments}

\bibliographystyle{abbrv}
\bibliography{etl-www2003}  % sigproc.bib is the name of the Bibliography in this case

%
%\balancecolumns
\appendix
%Appendix A
%\section{}
%\balancecolumns % GM July 2000

\end{document}
