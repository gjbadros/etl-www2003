%%% $Id$
%%% Copyright (C) 2002 Greg J. Badros <greg.badros@infospace.com>
%%
%
% This file should be compiled with V1.0 of "www2003-submission.cls"
%
% ----------------------------------------------------------------------------------------------------------------
% This .tex file (and associated .cls V1.0) produces:
%       1) NO Permission Statement
%       2) WWW'03-specific conference (location) information
%       3) The Copyright Line with ACM data
%       4) NO page numbers
%
% ---------------------------------------------------------------------------------------------------------------
% This .tex source is an example which *does* use
% the .bib file (from which the .bbl file % is produced).
% REMEMBER HOWEVER: After having produced the .bbl file,
% and prior to final submission, you *NEED* to 'insert'
% your .bbl file into your source .tex file so as to provide
% ONE 'self-contained' source file.
%

\documentclass{www2003-submission}
\usepackage{times}
\usepackage{moreverb}
\usepackage{floatflt}

\newcommand{\B}{\discretionary{}{}{}}
\newcommand{\smtt}{\small}
\newcommand{\smtexttt}[1]{{\small\texttt{#1}}}
\newcommand{\figref}[1]{Figure~\ref{#1}}
\newcommand{\tableref}[1]{Table~\ref{#1}}
\newcommand{\tm}{{\scriptsize $^{\mbox{tm}}$}}

\begin{document}
%
\title{The Extensible Templating Language: \\
       Improving Static Analysis of Markup Generation}

\numberofauthors{1}

\author{
%
% The command \alignauthor (no curly braces needed) should
% precede each author name, affiliation/snail-mail address and
% e-mail address. Additionally, tag each line of
% affiliation/address with \affaddr, and tag the
%% e-mail address with \email.
\alignauthor Greg J. Badros\\
       \affaddr{InfoSpace, Inc.}\\
       \affaddr{601 108th Ave. NE, Suite 1200}\\
       \affaddr{Bellevue, WA ~ USA}\\
       \email{greg.badros@infospace.com}
}
\date{10 October 2002}
\maketitle
\begin{abstract}
ETL....
\end{abstract}

% A category with only the three required fields
%% GREGB:FIXME::
\category{D.2}{Software}{Software Engineering}
\category{D.3}{Software}{Programming Languages}

% GREGB:FIXME:: any terms?
%\terms{}

\keywords{Static analysis, web server, templates, XML, XSLT, HTML.}

\section{Introduction}

Server-side dynamically-generated web pages have become increasingly
common and more complex.  The loosely-coupled client/\B{}server model
implied by the current breed of applications deployed via the World
Wide Web presents substantial new complexities for application
developers, and a variety of new programming languages and paradigms
attempt to address these novel problems.  One important characteristic
of this new world of software development is the need to generate HTML
or other markup languages to describe the client-side presentation and
behaviour.

The original server-side dynamic markup-generation capabilities of the
Web were defined by CGIs.\cite{CGI} For each request, the web server
maps HTTP request details into environment variables, command-line
arguments, and the standard file descriptors.  In particular, the
standard output written by that process was sent by the Web server as
the response to the client browser (instead of simply responding with
an unchanging file from disk).  Over time, various techniques arose to
reduce the cost of each request: extension mechanisms such as the
Apache module system~\cite{ApacheModules} and scripting languages
hosted by the Web server have increased performance by eliminating
process creation overhead.

The variety of server-side scripting is dominated by expressive and
flexible general-purpose programming languages such as
Java~\cite{Java}, Perl~\cite{Perl}, and PHP~\cite{PHP}.  Ironically,
the languages are then used in fairly restricted ways, often being
tied to a templating language (e.g., JSP~\cite{JSP} for Java or
HTML::Template~\cite{HTML-Template} for Perl).  For example, consider
the PHP template in \figref{php-books}.  The only constructs required
by the template are a) some means of populating a data model from the
back-end business logic (line 1); b) data-driven iteration (line
5-6); and c) simple expression evaluation to access parts of the data
model (lines 7 and 9). % GREGB:FIXME:: check these line numbers

\begin{figure}[htbp]
\begin{listing}{1}
<? $books = GetBooks(...); ?>
<html>
 <table>
  <tr>
   <? for ($i = 0; 
           $i < count($books); ++$i) { ?>
     <td><? print($books[$i][author]); ?>
         </td>
     <td><? print($books[$i][title]); ?>
         </td>
   <? } ?>
  </tr>
 </table>
</html>
\end{listing}%$
\caption{PHP template illustrating the simplicity of the programming
language constructs required by typical templates.
\label{fig-php-template}}
\end{figure}

Unfortunately, for most contemporary web application development
frameworks, the restrictions desired of the front-end templating layer
are imposed only by process and convention.  Template authors are
asked to limit themselves to a subset of the available features of a
language as a means of facilitating scalable, secure, well-behaved
systems.  Importantly, however, there is nothing inherent that
prevents template executions from, for example, making individual
database connections (which would impact performance) or maintaining
undesirable server-side state (which would impose additional
requirements on the load-balancing mechanism and reduce scalability).
These concerns are intensified because many development organizations
encourage a strong separation between the web developers and back-end
application developers. While web developers have substantial
expertise in HTML, user-interface design, and client-side scripting,
their responsibilities often do not include consideration of the
larger architecture.

Another substantial problem with existing templating technologies is
that they often involve the lexical mixing of two separate programming
paradigms.  JSP, for example, is implemented in terms of a
pre-processing rewrite of the template into an equivalent servlet.
Pre-processing approaches are convenient for programmer expressiveness
but have a huge hidden cost in complicating software engineering
analyses and tools that would otherwise help understand, maintain, and
evolve the complex systems~\cite{PCP3,EvilMacros,StroustropDnEChapterAboutCpp}.
The precise analysis of an arbitrary JSP template necessarily requires
full knowledge of the semantics of Java.

%% GIVE EXAMPLES OF VALUABLE ANALYSES!!!
\subsection{Analysis of dynamic web pages}

\subsection{Flexibility vs.\ analyzability}

\subsection{Outline of paper}


%%% 
\section{Background}

\subsection{Server-side web applications}

\subsection{Current templating strategies}

\subsection{Comparison of approaches}

%\begin{table}
%\centering
%\caption{Summary of existing templating approaches.}
%\begin{tabular}{|c|c|l|} \hline
%\\ \hline
%\hline\end{tabular}
%\end{table}


%%% 
\section{ETL: A better approach}


%%% 
\section{Benefits of ETL}

%%% 
\section{Implementation \& experience}

%%% 
\section{Related work}


%%% 
\section{Conclusions \& future work}


%ACKNOWLEDGMENTS are optional
\section{Acknowledgments}

\bibliographystyle{abbrv}
\bibliography{etl-www2003}  % sigproc.bib is the name of the Bibliography in this case

%
%\balancecolumns
\appendix
%Appendix A
%\section{}
%\balancecolumns % GM July 2000

\end{document}
